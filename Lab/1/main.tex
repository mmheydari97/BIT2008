\documentclass{homework}
\class{BIT2008A/ITEC2000A: Multimedia Data Management}
\author{}
\date{Fall 2022}
\title{Lab 1}

\graphicspath{{./pics/}}

\begin{document} \maketitle

\section{SQL Setup}
This is an easy and safe way to create an isolated environment to do the coding for this course. Please follow these steps:
\begin{enumerate}
    \item Download Oracle VirtualBox application for free, using \href{https://www.virtualbox.org/wiki/Downloads}{this link}.
    \fig{0.8}{vb.png}{VirtualBox download page.}{vb}
    The installation process is rather straight forward.
    
    \item Download Ubuntu image from \href{https://ubuntu.com/download/desktop}{this link}.
    \fig{0.8}{ubuntu.png}{Ubuntu dowload page.}{ubuntu}
    
    \item Run VirtualBox. Then click on New as shown in Fig.\ref{vm1}.
    \fig{0.8}{vm1.png}{New virtual machine.}{vm1}
    
    \item Specify the machine name and folder as shown in Fig.\ref{vm2}.
    \fig{0.5}{vm2.png}{VM location.}{vm2}

    \item Move Ubuntu image to the machine folder recently created at the directory that you specified.
    \item Assign a memory size on the green spectrum as shown in Fig.\ref{vm3}.
    \fig{0.5}{vm3.png}{Memory size.}{vm3}
    
    \item Stick to the defaults until you can specify the storage as shown in Fig.\ref{vm4}.
    \fig{0.5}{vm4.png}{Storage size.}{vm4}
    
    \item Click on the Settings tab as shown in Fig.\ref{vm5}.
    \fig{0.5}{vm5.png}{VM settings.}{vm5}
    
    \item Change these options to improve performance based on your system specifications as shown in Fig.\ref{vm6}.
    \fig{0.5}{vm6.png}{VM settings parameters.}{vm6}
    
    \item Press start and choose the disk image that you moved to the machine folder. Wait for the image to boot then choose "Try or install Ubuntu".
    
    \item Proceed with the installation. Choose extra software to be installed as shown in Fig.\ref{vm7}.
    \fig{0.5}{vm7.png}{VM extra software installation.}{vm7}
    
    \item After the installation is complete, power off the VM, remove the disk image as shown in Fig.\ref{vm8}, and then start the VM again.
    \fig{0.5}{vm8.png}{Remove disk image.}{vm8}
    
    \item Next, install PostgreSQL using \href{https://www.tecmint.com/install-postgresql-and-pgadmin-in-ubuntu/}{this link}.
    \item Now create your user based on the OS username so you have easier terminal access.
    \newpage
    \lstinputlisting[language=SQL, caption={psql commands.}, label=code]{code/code.tex}
    \item Finally, continue install pgAdmin4 web interface and try to connect to your database following \href{https://www.tecmint.com/install-postgresql-and-pgadmin-in-ubuntu/}{this link}.
    
\end{enumerate}



% \question Write down sets in order of containment.

% We pretend that equivalence classes are just numbers.
% \[
% 	\C \supset \R \supset \Q \supset \Z \supset \N \supset
% 	\P \not\supset (\GF[7] = \modulo[7])  \supset \{\nil\}
% \]

% \question Find roots of $x^2- 8x = 9$.
% \begin{align*}
% goigre & & \\
% guioshfg s& & \\
% \end{align*}

% We proceed by factoring,
% \begin{align*}
% 	x^2- 8x - 9     & = 9-9         &  & \text{Subtract 9 on both sides.}         \\
% 	x^2- x + 9x - 9 & = 0           &  & \text{Breaking the middle term.}         \\
% 	(x - 1)(x + 9)  & = 0           &  & \text{Pulling out common } (x - 1).      \\
% 	x               & \in \{1, -9\} &  & f(x)g(x) = 0 \Ra f(x) = 0 \vee g(x) = 0. \\
% \end{align*}

% \fig{0.3}{diagram.jpg}{Cipher wheels.}{wheel}

% \question Figure \ref{wheel} shows two cipher wheels. The left one is from Jeffrey Hoffstein, et al. \cite{hoffstein2008introduction} (pg. 3). Write a Python 3 program that uses it to encrypt: \texttt{FOUR SCORE AND SEVEN YEARS AGO}.

% \lstinputlisting[language=Python, caption={Python 3 implementing figure \ref{wheel} left wheel.}, label=gcd]{code/prog.py}

% We get: \texttt{KTZW XHTWJ FSI XJAJS DJFWX FLT}.

% % citations
% \bibliographystyle{plain}
% \bibliography{citations}

\end{document}