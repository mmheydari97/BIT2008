\documentclass{homework}
\class{BIT2008A/ITEC2000A: Multimedia Data Management}
\author{}
\date{Fall 2022}
\title{Lab 5}

\graphicspath{{./pics/}}
\usepackage{pdfpages}
\begin{document} \maketitle

\section{Python Tricks}
This tutorial aims to get you familiar with 11 Python features that make it easier to develop your programs. There are a couple of ways to run python. First, we can write our programs in (.py) files using code editors like Notepad or Visual Studio Code, and run them via the Python interpreter. If you have installed Anaconda, you can open Anaconda Navigator and use your favorite prompt.
\fig{1}{prompt.png}{Python prompts available in PowerShell and CMD.}{prompt}
To run your program, you need to locate the file and then write \textbf{\textit{python [path to .py file]}} in the terminal.
Secondly, There are interactive Python consoles to execute each line that you write in real-time. You can find the Qt Console shown in Fig. \ref{prompt} or type \textbf{\textit{ipython}} in your terminal if you have activated Anaconda there (Virtual Machine option). Finally, you can create a Jupyter Notebook to run your code, write instructions and comments, and present the results. Jupyter Notebooks enables you to create a mixture of text cells and code cells to report your Python projects more easily.

You can use google \href{https://colab.research.google.com/}{colab} to create and run Jupyter Notebooks online. Jupyter Notebooks can be converted to other formats such as (.pdf) or (.html) easily. The rest of this instruction is directly copied from the Jupyter Notebook that you are provided with.
\newpage

\includepdf[pages=-]{Lab5.pdf}

\end{document}