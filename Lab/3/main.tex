\documentclass{homework}
\class{BIT2008A/ITEC2000A: Multimedia Data Management}
\author{}
\date{Fall 2022}
\title{Lab 3}

\graphicspath{{./pics/}}

\begin{document} \maketitle

\section{Relational Algebra}
Figure \ref{rop} shows some of the commonly used relational algebra operators.
\fig{1}{op.png}{Relational algebra operators.}{rop}

\subsection*{Q1} Having the relation schema below for committee member management, please answer the questions below using relational algebra.
\begin{itemize}[label=]
    \item professor (\underline{prof\_name}, dept\_name)
    \item department (\underline{dept\_name}, building)
    \item committee (\underline{comm\_name}, \underline{prof\_name})
\end{itemize}

\begin{enumerate}[label=\alph*)]
    \item Project committee names having members from the "ECE" department.
    \item Project the name of professors sharing at least one committee with Prof. "Smith".
\end{enumerate}

\subsection*{Q2}
Having the relation schema below for medical records, please answer the questions below using relational algebra.
\begin{itemize}[label=]
    \item patient(\underline{p\_id}, p\_name, address)
    \item doctor(\underline{d\_id}, d\_name, hospital)
    \item medicine(\underline{m\_id}, m\_name)
    \item appointment(\underline{a\_id}, p\_id, d\_id, date)
    \item prescription(\underline{a\_id}, \underline{m\_id})
\end{itemize}
\begin{enumerate}[label=\alph*)]
    \item Show those medicines that have been listed at least once in the prescriptions of Dr. "Luis".
    \item Project the name and address of patients who have visited any doctors of "Garcia" hospital at least once.
    \item Project the name of medicines that have been used in at least one prescription.
    \item Assuming that all the names are unique, list the name of doctors who have visited another doctor. (In a scenario that doctor \textit{a} as a patient goes to doctor \textit{b}'s office, we want doctor \textit{b}'s name.)
\end{enumerate}

\section{Implementation}
\subsection*{Q3}
In this example, we deal with a Boat Reservation System. Please follow the step-by-step instructions using the code provided in instructions.sql.
\begin{enumerate}[label=\alph*)]
    \item Make sure you are on the right database using \textit{current\_database()} function.
    \item Create sufficient tables for the schema below:
    \begin{itemize}[label=]
        \item sailor(\underline{s\_id}, s\_name, age, rating)
        \item boat(\underline{b\_id}, b\_name, color)
        \item reserve(\underline{s\_id}, \underline{b\_id}, \underline{r\_date})
    \end{itemize}
    For convenience, the code to create a table for Sailors is provided. Please 
    create the tables one by one.
    
    \item Insert sailors and boats instances as Tables \ref{tab:1} and \ref{tab:2} below. The command to insert the first row is provided.
    \begin{table}[!ht]
        \centering
        \begin{tabular}{|c|c|c|c|}
             \hline
             s\_id& s\_name& age& rating\\
             \hline
             1& James& 45& 10\\
             \hline
             2& Mary& 36& 5\\
             \hline
             3& Robert& 50& 9\\
             \hline
             4& Patricia& 34& 8\\
             \hline
             5& John& 24& 2\\
             \hline
             6& Jennifer& 20& 2\\
             \hline
             7& Michael& 20& 10\\
             \hline
             8& Linda& 35& 3\\
             \hline
             9& David& 58& 10\\
             \hline
             10& Elizabeth& 55& 5\\
             \hline
        \end{tabular}
        \caption{Sailors}
        \label{tab:1}

        \begin{tabular}{|c|c|c|}
             \hline
             b\_id& b\_name& color\\
             \hline
             1& Serendipity& Blue\\
             \hline
             2& Imagination& Red\\
             \hline
             3& Liberty& Blue\\
             \hline
             4& Wanderlust& Green\\
             \hline
             5& Gale& Blue\\
             \hline
             6& Zephyr& Green\\
             \hline
        \end{tabular}
        \caption{Boats}
        \label{tab:2}
    \end{table}

    \item Make sure that the rows are inserted correctly by selecting all the values from each table.

    \item Insert reservation records based on the code provided.
    \item List the IDs of sailors who had at least one trip on the boat with ID 3. You only need \textit{reserve} table for this query.
    \item Now list the name of the sailors that you listed in previous query. Their names are listed in \textit{sailor} table. Use your previous query to check the WHERE condition.
    \item We have two new Sailors who have not been on a trip yet. List their names.
    \item List all the sailor IDs who have been on a trip with a red or green boat. Use keyword DISTINCT to remove duplicates from your result.
    \item We have a sailor who likes red boats and does not like green boats. Write a query to find their name. You may find the keyword EXCEPT handy to implement set difference. 
    
\end{enumerate}


% \question Write down sets in order of containment.

% We pretend that equivalence classes are just numbers.
% \[
% 	\C \supset \R \supset \Q \supset \Z \supset \N \supset
% 	\P \not\supset (\GF[7] = \modulo[7])  \supset \{\nil\}
% \]

% \question Find roots of $x^2- 8x = 9$.
% \begin{align*}
% goigre & & \\
% guioshfg s& & \\
% \end{align*}

% We proceed by factoring,
% \begin{align*}
% 	x^2- 8x - 9     & = 9-9         &  & \text{Subtract 9 on both sides.}         \\
% 	x^2- x + 9x - 9 & = 0           &  & \text{Breaking the middle term.}         \\
% 	(x - 1)(x + 9)  & = 0           &  & \text{Pulling out common } (x - 1).      \\
% 	x               & \in \{1, -9\} &  & f(x)g(x) = 0 \Ra f(x) = 0 \vee g(x) = 0. \\
% \end{align*}

% \fig{0.3}{diagram.jpg}{Cipher wheels.}{wheel}

% \question Figure \ref{wheel} shows two cipher wheels. The left one is from Jeffrey Hoffstein, et al. \cite{hoffstein2008introduction} (pg. 3). Write a Python 3 program that uses it to encrypt: \texttt{FOUR SCORE AND SEVEN YEARS AGO}.

% \lstinputlisting[language=Python, caption={Python 3 implementing figure \ref{wheel} left wheel.}, label=gcd]{code/prog.py}

% We get: \texttt{KTZW XHTWJ FSI XJAJS DJFWX FLT}.

% % citations
% \bibliographystyle{plain}
% \bibliography{citations}

\end{document}