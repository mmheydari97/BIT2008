\documentclass{homework}
\class{BIT2008A/ITEC2000A: Multimedia Data Management}
\author{}
\date{Fall 2022}
\title{Lab 3}

\graphicspath{{./pics/}}

\begin{document} \maketitle

\section{Relational Algebra}


\subsection*{Q1}
\begin{itemize}[label=]
    \item professor (\underline{prof\_name}, dept\_name)
    \item department (\underline{dept\_name}, building)
    \item committee (\underline{comm\_name}, \underline{prof\_name})
\end{itemize}

\begin{enumerate}[label=\alph*)]
    \item
    \[\Pi _{comm\_name}\left(committee\:\bowtie \:\left(\Pi _{prof\_name}\left(\sigma _{dept\_name='ECE'}\left(professor\right)\right)\right)\right)\]
    \item 
    \[\Pi _{prof\_name}\left(committee\:\bowtie \:\Pi _{comm\_name}\left(\sigma _{prof\_name='Smith'}\left(committee\right)\right)\right)\]
\end{enumerate}

\subsection*{Q2}
Having the relation schema below for medical records, please answer the questions below using relational algebra.
\begin{itemize}[label=]
    \item patient(\underline{p\_id}, p\_name, address)
    \item doctor(\underline{d\_id}, d\_name, hospital)
    \item medicine(\underline{m\_id}, m\_name)
    \item appointment(\underline{a\_id}, p\_id, d\_id, date)
    \item prescription(\underline{a\_id}, \underline{m\_id})
\end{itemize}
\begin{enumerate}[label=\alph*)]
    \item 
    \[medicine\:\bowtie \:\Pi _{m\_id}\left(prescription\:\bowtie \:\Pi _{a\_id}\left(appointment\:\bowtie \:\Pi _{d\_id}\left(\sigma _{d\_name='Luis'}\left(doctor\right)\right)\right)\right)\]
    \item 
    \[\Pi _{p\_name,\:address}\left(patient\:\bowtie \:\Pi _{p\_id}\left(appointment\:\bowtie \:\Pi _{d\_id}\left(\sigma _{hospital='Garcia'}\left(doctor\right)\right)\right)\right)\]
    \item
    \[\Pi _{m\_name}\left(medicine\:\bowtie \:\Pi _{m\_id}\left(prescription\right)\right)\]
    \item 
    \[\Pi _{d\_name}\left(doctor\:\bowtie \:\Pi _{d\_id}\left(appointment\:\bowtie \:\Pi _{p\_id}\left(\sigma _{p\_name=d\_name}\left(patient\:\times doctor\right)\right)\right)\right)\]
\end{enumerate}



% \question Write down sets in order of containment.

% We pretend that equivalence classes are just numbers.
% \[
% 	\C \supset \R \supset \Q \supset \Z \supset \N \supset
% 	\P \not\supset (\GF[7] = \modulo[7])  \supset \{\nil\}
% \]

% \question Find roots of $x^2- 8x = 9$.
% \begin{align*}
% goigre & & \\
% guioshfg s& & \\
% \end{align*}

% We proceed by factoring,
% \begin{align*}
% 	x^2- 8x - 9     & = 9-9         &  & \text{Subtract 9 on both sides.}         \\
% 	x^2- x + 9x - 9 & = 0           &  & \text{Breaking the middle term.}         \\
% 	(x - 1)(x + 9)  & = 0           &  & \text{Pulling out common } (x - 1).      \\
% 	x               & \in \{1, -9\} &  & f(x)g(x) = 0 \Ra f(x) = 0 \vee g(x) = 0. \\
% \end{align*}

% \fig{0.3}{diagram.jpg}{Cipher wheels.}{wheel}

% \question Figure \ref{wheel} shows two cipher wheels. The left one is from Jeffrey Hoffstein, et al. \cite{hoffstein2008introduction} (pg. 3). Write a Python 3 program that uses it to encrypt: \texttt{FOUR SCORE AND SEVEN YEARS AGO}.

% \lstinputlisting[language=Python, caption={Python 3 implementing figure \ref{wheel} left wheel.}, label=gcd]{code/prog.py}

% We get: \texttt{KTZW XHTWJ FSI XJAJS DJFWX FLT}.

% % citations
% \bibliographystyle{plain}
% \bibliography{citations}

\end{document}