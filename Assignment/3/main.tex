\documentclass{homework}
\class{BIT2008A/ITEC2000A: Multimedia Data Management}
\author{}
\date{Fall 2022}
\title{Assignment 3}

\graphicspath{{./pics/}}

\begin{document} \maketitle

\section*{Python and Django}
\question Please explain what a design pattern is, then elaborate on the MTV design pattern used by Django. (10 points)


\question Please explain the difference in debugging between a programming language that uses an interpreter and a compiler. (5 points)

\question Name 4 collection data types in Python explaining their differences. (10 points)

\question Please explain 4 pillars of object-oriented programming and their implementation in Python. (20 points)


\section*{Implementation}
In this part, we create a bookkeeping application. Please make sure to implement the following steps: (55 points)
\begin{enumerate}[label=\roman*.]
    \item Create a Django project and make sure it uses PostgreSQL as the database.
    \item Add a main app to your project.
    \item Implement this schema in your app:
    \begin{itemize}
        \item Book(title, author, publish\_year, edition)
        \item Member(name, address, phone\_number)
        \item Booking(member, book, expiration\_date)
    \end{itemize}
    It is up to you to choose the primary keys.
    \item Make it possible to create instances of your models inside the admin panel.
    \item Implement following functionalities:
    \begin{itemize}
        \item The root of your application should show a default welcome message via \textit{HttpResponse}.
        \item There should be three \textit{ListView}s to list all of the records in each model.
        \item Your booking template should have a button that redirects to a form.
        \item Use \textit{CreateView} to implement the form mentioned above that handles new bookings.
        \item After the booking is done you should redirect the user to the list of bookings.
        \item If a book or member is removed, the corresponding records in the booking table should be removed.
        \item Imagine we have unlimited inventory for the books listed.
        \item Each member should have at most one active booking.
        \item Your app should not take invalid expiration\_dates. That includes dates that are already gone, or expiration dates sooner than the existing booking for a member.
        \item However, you should make it possible to extend an active booking.
        \item Extended bookings should not take a new row, but you should update the expiration date of existing row.
        \item You are not expected to implement other details about the business logic.
    \end{itemize}
\end{enumerate}
\newpage
\section*{Description}
\begin{enumerate}
    \item The due date for this assignment is November 28th, 11:55 PM. Late submission policy can be found on the course outline.
    \item You are expected to submit your solution for all of the assignments.
    \item Please include your answer to the first part of the assignment (Q1 to Q4) in a PDF file. It can be handwritten, but make sure it is easily readable.
    \item Please put your code implementation in a folder, without the database, cached, or migration files. Export your database to a script and name it \textit{table.sql}.
    \item Please upload your submission as Lastname\_Firstname\_StudentID.zip on Brightspace.
\end{enumerate}

\end{document}